\documentclass[10pt]{article}
\usepackage[fleqn]{amsmath}
\usepackage{parskip}
\usepackage[a4paper,top=2cm,bottom=2cm,left=3cm,right=3cm,marginparwidth=1.75cm]{geometry}

\setlength{\parindent}{0pt}

\begin{document}

\date{}
\author{Alessio Di Casola (mat. 239490), Marco Sterlini (mat. 236865)}

\title{Assignment 01: Comparison between different controllers}

\maketitle

\section{Answers}
\subsection{Question 1}
The different implementations of the IC controller differ for some characteristics.
We imposed stiffness and damping matrices as diagonal with constant stiffness values and damping values dynamically computed using the simulator mass matrix imposing a unitary damping ratio. 

Every controller manages to converge to the desired end effector position except for the third one (Exact version). The first one (Simplified version) manifests a continuous movements of the arm even though the end effector remains still, this because we didn't minimize the null-space of the jacobian pseudo-inverse yet.

In the second (Simplified version + postural task) and fourth (Exact version + postural task) controller we don't notice any macroscopic difference from the simulations, the tracking errors are different: \dots and they converge to an equilibrium in different times: \dots

When we introduced the friction we noticed that the controller couldn't manage to always make the robot reach the desired e.e. position. This can be physically explained because the control torque is proportional (but not only) to the error position, once it becomes small enough the resulting control torque becomes equal and opposite to the sum of the friction torque and gravity compensation resulting in a torque equilibrium (fig. \ref{equilibrium}) without ever reaching the desired position (fig. \ref{friction_explained}).

Here are the average tracking errors of the controllers:

\begin{table}[h]
    \begin{tabular}{lllll}
    \cline{1-4}
    \multicolumn{1}{|l|}{Simplified} & \multicolumn{1}{l|}{Simplified + postural task} & \multicolumn{1}{l|}{Exact} & \multicolumn{1}{l|}{Exact + postural task} &  \\ \cline{1-4}
    \multicolumn{1}{|l|}{} & \multicolumn{1}{l|}{} & \multicolumn{1}{l|}{\textbf{Undefined}} & \multicolumn{1}{l|}{} &  \\ \cline{1-4}
                           &                       &                       &                       &  \\
                           &                       &                       &                       & 
    \end{tabular}
\end{table}

\subsection{Question 2}

In the exact controller we see that the torques diverge and hence we can't finish the simulation, in the friction case scenario instead it manages to converge. This is explained by the behavior of the non controlled joint space: in the first case it diverges with increasing velocities and the $\dot J \dot q$ term diverges leading to the end of simulation, in the second case the uncontrolled space is mechanically stable due to the addition of friction. Hence we manage to reach an overall converging control law only by controlling the main task (fig. \ref{exact_diverging}).

The other controls manage to stabilize the desired position because they are imbued with the postural task, the first case instead, even though it doesn't have the postural task, doesn't diverge because we made the assumption that joint velocities won't be high, doing so we don't multiply the term $\dot J \dot q$ by $J^T$ and we don't encounter the numerical divergence (fig. \ref{comparison}) Non linear effects are portrayed by $h$ which is simply added to the control law remaining reasonably bounded.
\subsection{Question 3}
\subsection{Question 4}
\subsection{Question 5}

\pagebreak

\section{Plots}

Notes on comparison plot:

On the left we can see the values of $h$, $\mu$ and $\dot J \dot q$ for some step of the simplified controller, on the right the equivalent for the exact controller moments before divergence. We can see how with increasing velocities the terms tend to diverge in magnitude

\end{document}