\documentclass[12pt]{article}
\usepackage[fleqn]{amsmath}
\usepackage{parskip}
\usepackage[a4paper,top=2cm,bottom=2cm,left=3cm,right=3cm,marginparwidth=1.75cm]{geometry}

\setlength{\parindent}{0pt}

\begin{document}

\date{}
\author{Alessio Di Casola (mat. 239490), Marco Sterlini (mat. 236865)}

\title{Assignment 01: Comparison between different controllers}

\maketitle

\section{Answers}
\subsection{Question 1}
The different implementations of the IC controller differ for some characteristics.
We imposed stiffness and damping matrices as diagonal with constant stiffness values and damping values dynamically computed using the simulator mass matrix imposing a unitary damping ratio. 

Every controller manages to converge to the desired end effector position except for the third one (Exact version). The first one (Simplified version) manifests a continuous movements of the arm even though the end effector remains still, this because we didn't minimize the null-space of the jacobian pseudo-inverse yet.

In the second (Simplified version + postural task) and fourth (Exact version + postural task) controller we don't notice any macroscopic difference from the simulations, the tracking errors are different: \dots and they converge to an equilibrium in different times: \dots

When we introduced the friction we noticed that the controller couldn't manage to always make the robot reach the desired e.e. position.

As a result here are the average tracking errors of the controllers:

\begin{table}[h]
    \begin{tabular}{lllll}
    \cline{1-4}
    \multicolumn{1}{|l|}{Simplified} & \multicolumn{1}{l|}{Simplified + postural task} & \multicolumn{1}{l|}{Exact} & \multicolumn{1}{l|}{Exact + postural task} &  \\ \cline{1-4}
    \multicolumn{1}{|l|}{} & \multicolumn{1}{l|}{} & \multicolumn{1}{l|}{\textbf{Undefined}} & \multicolumn{1}{l|}{} &  \\ \cline{1-4}
                           &                       &                       &                       &  \\
                           &                       &                       &                       & 
    \end{tabular}
\end{table}

\subsection{Question 2}
\subsection{Question 3}
\subsection{Question 4}
\subsection{Question 5}

\section{Plots}

\end{document}